\section{Заключение}
В рамках данной работы был составлен обзор существующих решения, разработана архитектура рантайма кадровых графов, написана программная реализация предлагаемой архитектуры, произведена интеграция разработанного программного решения в коммерческие проекты, вопрос управления памяти был сведён к не рассмотренной ранее в литературе задаче дискретной оптимизации CDSA, предложена схема приближённого решения этой задачи, и проанализированы качество и скорость работы предложенной схемы.
Предложенное решение управления памятью позволяет сэкономить близкое к оптимальному количество памяти на данных, приближенных к реальным.

В рамках дальнейших исследований, во-первых, планируется проверка следующей гипотезы.
Ясно, что для некоторых промежуточных графов, выбранная топологическая сортировка влияет на величину $LOAD$.
Из этого резонно поставить вопрос о нахождении оптимальной топологической сортировки, минимизирующей $LOAD$, а значит и позволяющей уменьшить размер итогового ответа $makespan$.
Во-вторых, имеет смысл обобщение алгоритмов решения DSA предложенных в работах Йордана Гергова \cite{gergov_approximation_1996, gergov_algorithms_1999} на случай CDSA.
В-третьих, крупной, но малоизученной областью развития кадровых графов является автоматическая параллелизация вычислений как на CPU, так и на GPU.
Наконец, объединяя все предыдущие пункты, дальнейший интерес представляет возможность адаптации алгоритмов оптимизации кадрового графа под специфику конкретных устройств. Так, на TBDR-GPU для производительности графа в целом лучше группировать барьеры воедино (см. \cite{GraniteBlogPost}), сортируя при этом граф соответствующим образом, в то время как на дискретных GPU барьеры, как правило, слабее влияют на производительность, что может позволить выбрать иную сортировку графа, способствующую уменьшению $LOAD$. Параллельное исполнение графа также зависит от специфики устройства, количества ядер, количества модулей обработки списков команд на GPU, из чего можно полгать, что, вообще говоря, для различных устройств может быть оптимальным различное параллельное расписание исполнения вершин.

Кадровые графы "--- техника, совсем недавно получившая распространение в области компьютерной графики, но в перспективе способная улучшить все аспекты разработки приложений реального времени: потребление памяти, скорость работы приложения, и даже удобство разработки. Автор считает, что дальнейшие исследования, а также тяжкий труд по реализации теоретических идей на практике, позволят раскрыть этот потенциал и однажды сделать разработку высокопроизводительных приложений реального времени простым и приятным занятием.
