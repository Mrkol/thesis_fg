\section{Заключение}
В рамках данной работы была разработана архитектура рантайма кадровых графов, написана программная реализация предлагаемой архитектуры, произведена интеграция разработанного программного решения в коммерческие проекты, вопрос управления памяти был сведён к не рассмотренной ранее в литературе задаче дискретной оптимизации CDSA, и наконец предложена схема приближённого решения этой задачи.
Предложенное решение управления памятью позволяет сэкономить близкое к оптимальному количество памяти на данных, приближенных к реальным.

В рамках дальнейших исследований, во-первых, планируется уделить больше ресурсов проверке гипотезы о влиянии стратегии выбора топологической сортировки на значение $LOAD$ из раздела \ref{negativeResults}.
Во-вторых имеет смысл обобщение алгоритмов решения DSA предложенных в работах Йордана Гергова \cite{gergov_approximation_1996, gergov_algorithms_1999} на случай CDSA.
В-третьих, крупной областью развития кадровых графов является автоматическая параллелизация вычислений как на CPU, так и на GPU, выходящая за рамки данной работы.
