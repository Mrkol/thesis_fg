\section{Предложенное решение}
\subsection{Общая архитектура}
Кадровый граф в первую очередь состоит из вершин, каждая из которых в свою очередь состоит из имени и функций \textit{объявления} и \textit{запуска}, задаваемых пользователем рантайма.
Как следует из названия, функция запуска вызывается рантаймом в момент исполнения кадрового графа для записи командных списков, а функция объявления вызывается при перекомпиляции графа чтобы предоставить пользователю возможность установить следующие скрытые свойства вершины:
\begin{itemize}
    \item ресурсные и вершинные зависимости,
    \item назначения ресурсов,
    \item режим \textit{мультиплексирования} вершины,
    \item флаг наличия побочных эффектов,
    \item требуемое для запуска состояние драйвера.
\end{itemize}
Пользователь может в любой момент создать, удалить или изменить вершину, на что рантайм автоматически отреагирует на ближайшем запуске кадрового графа.
Все существующие в программе вершины вершины организуются в единую структуру называемую кадровым графом, но в строгом математическом смысле кадровым графом не являющуюся.

Каждый кадр приложения пользователь даёт команду рантайму запустить текущий граф.
Перед тем как исполнить эту команду, рантайм проверяет не изменился ли граф с прошлого запуска, и в случае изменения начинает процесс \textit{компиляции графа}, состоящий из следующих пунктов:
\begin{itemize}
    \item объявление вершин,
    \item генерация промежуточного представления,
    \item построение расписания вершин,
    \item построение расписания ресурсов,
    \item активация \textit{истории ресурсов}.
\end{itemize}
Помимо вершин, кадровый граф содержит в себе некоторую вспомогательную информацию, при изменении которой нет нужды полностью перекомпилировать граф, поэтому процесс компиляции является инкрементальным. Так, например, при изменении разрешении окна, процесс перекомпиляции графа начнётся с предпоследнего пункта. Первый этап компиляции же запускается всякий раз когда пользователь изменил множество вершин, и именно в рамках этого этапа запускаются функции объявления изменившихся вершин.

Как уже было сказано выше, данные, изначально задаваемые пользователем рантайма, строго говоря не являются графом, а лишь схемой получения графа \textit{промежуточного представления}, с которым и работает остальной рантайм. \todo{диаграмма?}
Данные, полученные от пользователя, могут быть невалидными и противоречивыми, поэтому в процессе получения промежуточного представления также идёт валидация, и гарантируется, что получившийся граф будет удовлетворять следующему определению.
\textit{Граф промежуточного представления} -- четвёрка $(V, E, R, U)$, где $(V, E)$ -- вершины и рёбра ориентированного ациклического графа, $R$ -- множество ресурсов, а функция $U : V \to 2^R$ задаёт множество используемых каждой вершиной ресурсов. При этом требуется, чтобы для любого $r\in R$ подграф $(V, E)$ индуцированный множеством $\left\{v \in V \middle| r \in U(v)\right\}$ был непустым слабо-связным графом с ровно одной вершиной с входной степенью $0$.

На этапе построения расписания вершин строится топологическая сортировка промежуточного графа. В теории, именно на этом этапе может строится многопоточное расписание запуска функций исполнения вершин, а также могут выбираться разные очереди GPU для отправки команд исполняемыми вершинами. Однако, на практике, работа в этом направлении требует крупных предварительных вложений времени в сугубо инженерные вопросы, что не было предусмотрено в рамках бюджета данной работы.

Эти и оставшиеся этапы более подробно рассмотрены в следующих подразделах.

\subsection{Зависимости вершин}
\todo{тут про create-read-modify-rename, опциональность, pruning}

\subsection{Автоматические разрешения текстур}

\subsection{Мультиплексирование}
\todo{тут про всякий VR и 4К скриншоты}
