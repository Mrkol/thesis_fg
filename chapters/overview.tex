\section{Обзор существующих работ}
\subsection{Рантаймы кадровых графов}
Переходя к обзору различных имплементаций рантайма кадровых графов стоит отметить, что далеко не все компании готовы рассказывать о используемых в их разработках технологиях. Список, приведённый ниже, был составлен основываясь на открытых источниках информации и не претендует на полноту.
\subsubsection*{Frostbite}
Первыми идею организации архитектуры рендеринга в приложениях реального времени через вычислительные графы предложили разработчики движка Frostbite в 2017 году \cite{FrostbiteGdcTalk}.
\textit{Кадровый граф} позволил им сделать ядро модуля рендеринга расширяемым, упростил работу с асинхронным вычислениями общего назначения на GPU, автоматизировал работу со специализированными видами оперативной видеопамяти на игровых консолях, а также сэкономил большое количество обычной видеопамяти.
В силу проприетарности движка не известно, насколько широкий класс сценариев использования ресурсов она поддерживает.
В качестве схемы аллокации ресурсов же был взят обычный онлайн-аллокатор, располагающий в заранее выделенном крупном участке памяти ресурсы по мере необходимости.
Автоматическая расстановка барьеров на 2017 год не поддерживалась.

\subsubsection*{Halcyon}
Далее, в 2019 году, компания EA представила \cite{HalcyonRapidInnovationTalk} новый экспериментальный движок Halcyon, обобщающий идею кадрового графа до \textit{графа рендеринга}.
Как следует из названия, это обобщение позволяет организовывать в виде графа вычислений не только процесс рендеринга самого кадра, но и рендеринг различных вспомогательных изображений, например кубических карт для некоторых техник глобального освещения, изображений импосторов, или различных иконок в приложении.
Более того, граф рендеринга может состоять из нескольких подграфов, запускающихся с разной частотой, например подграф, вычисляющий тени от солнца, может запускаться только при значительном изменении положения солнца на небе.
В отличии от Frostbite, граф Halcyon поддерживает автоматическую расстановку барьеров.
Также на выступлении отмечается, что изначально граф составлялся явной композицией вершин и подграфов, но в итоге разработчики пришли к дизайну с автоматической композицией вершин на основе глобально видимых имён ресурсов.
Про алгоритм аллокации ресурсов и поддержку переживающих границу кадра ресурсов публично доступной информации нет.

Наконец, граф рендеринга Halcyon способен в автоматическом режиме масштабироваться на несколько аппаратных графических ускорителей, и даже на несколько компьютеров.
Поддержка такого функционала весьма сложна и говорит об экспериментальности этой разработки, так как на практике такая масштабируемость редко применима.

\subsubsection*{Unreal Engine}
Начиная с версии 4.22 в Unreal Engine начал переходить на рендеринг через систему "Render Dependency Graph" \cite{UERenderDependencyGraph}, представляющую собой граф рендеринга.
Однако даже в 5й версии движка эта система использует жадную он-лайн стратегию аллокации ресурсов, хоть и поддерживает большое количество важного функционала: автоматизацию асинхронных вычислений на GPU, расстановку барьеров и параллелизацию исполнения графа.

\subsubsection*{Unity}
Разработчики движка Unity, следуя общему направлению индустрии, в 2018 году перевели архитектуру рендеринга на подход вычислительных графов \cite{UnityRenderingArchitectureTalk}.
Однако среди прочих проприетарных движков про граф рендеринга Unity известно, пожалуй, меньше всего.
Отметить стоит лишь наличие интеграции между нативным и скриптовым кодом рендеринга, позволяющей сократить время итерации при разработке приложений.

\subsubsection*{Anvil}
Движок Anvil компании Ubisoft с переходом на DirectX12 тоже начал выделять подсистему зависимостей ресурсов, хоть и не называя её графом кадра \cite{DX12CaseStudies, AnvilDx12LessonsLearned}.
Согласно выступлениям на GDC, эта система поддерживает многий функционал уже упомянутых: переиспользует память ресурсов, автоматически расставляет барьеры, автоматизирует асинхронные вычисления.
Но как и в случае Unity, детали устройства отсутствуют в публичном доступе.

\subsubsection*{Render Pipeline Shaders}
Выпущенная в открытый доступ \cite{RPSgithub} в декабре 2022 года библиотека Render Pipeline Shaders компании AMD в своём составе имеет комплексное решение для построение кадровых графов \cite{RPSpost}.
Эта библиотека полностью скрывает от пользователя управление транзиентными ресурсами посредством предметно-ориентированного языка, автоматически переиспользуя ресурсы, расставляя барьеры и клонируя объявленные единожды вершины графа.
\todo{ПРОЧИТАТЬ ИСХОДНИКИ И НАПИСАТЬ ЕШО}

\subsubsection*{Granite}
Наконец, стоит упомянуть о существовании многих любительских проектов по написанию обобщённой библиотеки кадровых графов. Большинство из них находятся на стадии зарождения и не заслуживают подробного рассмотрения.
Исключеним является проект Granite \cite{GraniteBlogPost}, в рамках которого разработан кадровый граф адаптированный для использования на мобильных устройствах посредством API Vulkan.
Кадровый граф Granite автоматически расставляет примитивы синхронизации, группирует ноды в рендер-пассы \cite[раздел~8]{VulkanSpec}, оптимизирует порядок исполнения вершин с точки зрения минимизации накладных расходов на синхронизацию, а также переиспользует память, хоть и при помощи жадного алгоритма аллокации.

\subsection{Аллокация ресурсов}
Задача поиска расписания аллокации ресурсов в графе кадра в своей простейшей формулировке является классической сильно NP-сложной \cite{DSAnpcomplete} задачей \textit{динамической аллокации памяти} (dynamic storage allocation, DSA \cite[с. 226]{10.5555/574848}).
У этой задачи существует две интерпретации, он-лайн и офф-лайн.
Первая подразумевает обработку разнесённых во времени запросов на аллокацию и деаллокацию ресурсов, иначе говоря, решения об адресах ресурсов в памяти необходимо принимать в порядке времён появления ресурсов.
Этот частный случай часто встречается в операционных системах и рантаймах языков программирования.
Вторая же интерпретация подразумевает наличие заранее известных времён жизни всех ресурсов.
В рамках данной работы нас интересует именно офф-лайн интерпретация, поэтому, в отсутствие уточнения, под задачей о динамической аллокации памяти мы будем подразумевать именно её.

Одним из первых полиномиальных алгоритмов предложенных для решения задачи DSA является алгоритм First-Fit \cite{chrobak_packing_1988}, работающий, как было вскоре доказано Кирстедом, с константной ошибкой не более чем в 80 раз \cite{kierstead_linearity_1988}.
Тремя годами позже Кирстед представил алгоритм с ошибкой не более чем в 6 раз\cite{kierstead_polynomial_1991}.
Эти и другие ранние работы объединяет общий подход сведения DSA к частному случаю с единичным размером всех ресурсов, эквивалентному покраске интервального графа, и последующим применением он-лайн алгоритма покраски.
Через несколько лет Йордан Гергов, отказавшись от сведения к интервальным графам, смог понизить верхнюю оценку минимальной возможной ошибки до 5 \cite{gergov_approximation_1996}, а в последствии и до 3\cite{gergov_algorithms_1999}.
Наконец, наилучший на данный момент результат был получен исследователями из AT\&T Labs совместно с коллегой из Ecole Polytechnique \cite{buchsbaum_opt_2003}: полиномиальный алгоритм, для любого заранее выбранного $\varepsilon$ дающий $(2+\varepsilon)$-приблизительное решение DSA.
Более того, для некоторых частных случаев авторы предоставляют приближённую схему полиномиального времени (то есть $(1+\varepsilon)$-приближение).
Из них в рамках графа кадра особо интересна схема для случая ресурсов, размер которых ограничен сверху константой $h_{max}$.
Однако практичность представленных алгоритмов в рамках приложений реального времени является открытым вопросом в силу их высокой сложности (TODO: оценить асимптотику по мастер-теореме).

Похожая задача возникает в области оперирования морских контейнерных терминалов.
С ростом сложности и нагруженности глобальных транспортных цепочек, прикладные задачи оперирования верфей стали слишком сложны для интуитивного их решения.
В связи с этим за последние несколько десятилетий было сформулировано и в той или иной степени решено множество вариаций \textit{задачи об аллокации верфи}, покрывающих широкий спектр прикладных задач логистики.
Так как расписания прибытия кораблей обычно известно портам заранее, офф-лайн задача динамической аллокации памяти является частным случаем одной из формулировок этой задачи, а именно вариации классифицируемой в обзорной статье Бирвирта и Мизла \cite{BIERWIRTH2010615} как $cont|dyn|fix|max(res)$.
Именно из-за этого задача об аллокации верфи представляет интерес в рамках данной работы.
\todo{Может нафиг эти верфи?}

Одним из первых интересующую нас формулировку задачи об аллокации верфи рассмотрел в своей статье Эндрю Лим \cite{LIM1998105}.
Ресурсы, имеющие фиксированные и известные размер и времена аллокации и деаллокации, могут быть рассмотрены как корабли с соответствующей длинной, временем прибытия и временем отплытия, а тип используемой видеопамяти как секция верфи.
Задача нахождения минимальной длинны всех секций верфи и точек прибытия всех кораблей аналогична нахождению минимального необходимого объёма памяти и локаций всех ресурсов в этой памяти.
Однако в отличии от рассматриваемой Лимом задачи, ресурсы не накладывают требований на отступ между друг другом и началом или концом верфи, зато требуют определённого выравнивания их начала в памяти.
Впрочем, последние условие достаточно легко сводится к первому.

Однако в данной работе рассматривается более общая формулировка задачи об аллокации ресурсов, позволяющая также переиспользовать память ресурсов, используемых темпоральными алгоритмами, и, насколько известно автору, не рассматривавшаяся ранее в литературе.
