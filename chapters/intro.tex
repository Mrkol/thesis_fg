\section{Введение}

\subsection{Аллокация ресурсов}
В процессе вычисления картинки одного кадра любое нетривиальное приложение использует \textit{транзиентные ресурсы} -- промежуточные хранилища данных, содержимое которых не требуется после окончания вычисления кадра. Основная отличительная черта рассматриваемого подхода заключается в известности всей информации о транзиентных ресурсах заранее, что позволяет управлять ими более эффективно. Более того, наш подход позволяет добиться в определённом смысле оптимальной работы с такими ресурсами, как будет видно дальше. Однако обязательным пререквезитом для эффективной аллокации ресурсов является использование современного графического API, предоставляющего возможность ручного управления видеопамятью. До появления подобных API большая часть приложений использовало один из следующих наивных подходов к управлению ресурсами.

Самый простым подходом является выделение и освобождение транзиентных ресурсов по ходу их нужды при помощи соответствующих вызовов графического API. Этот подход сильно похож к управлению памятью объектов в системных языках программирования: драйвер операционной системы содержит аллокатор, на который пользователь перекладывает обязанность управления памятью и другими ресурсами GPU, аналогично куче в языке C. Системный аллокатор переиспользует освободившуюся память, тем самым достигая низкого её потребления. Однако такой подход не масштабируется на более сложные приложения. (фрагментация, отложенное удаление, рефкаунтинг, етц)

Альтернативным подходом служит отказ от переиспользования памяти. Все транзиентные ресурсы создаются заранее и не удаляются в ходе работы приложения. ...

Наконец, наиболее практичным подходом является пулинг ресурсов. ...
