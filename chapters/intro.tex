\section{Введение}
С самого появления области интерактивных приложений реального времени с трёхмерной графикой, главной задачей является максимизация качества картинки без потери производительности.
Так, основным бизнес"=требованием от индустрий кино, игр, виртуальных симуляторов, промышленной визуализации, является реализм итогового изображения.
С целью удовлетворения этих требований было проведено множество академических исследований различных методов визуализации, разработаны новые алгоритмы и техники отрисовки сцен.
Параллельно с этим не стояла на месте и область потребительской и промышленной техники.
В настоящий день не редка ситуация, когда одно и то же приложение должно запускаться на целом ряде различных устройств: персональных компьютерах абсолютно разной комплектации, стационарных и портативных игровых консолях, мобильных устройствах различных производителей и даже HMD-устройствах виртуальной и дополненной реальности\footnote{Устройства, закрепляемые на голове пользователя, от англ. head mounted device.}.

Вопрос потребления видеопамяти такими приложениями является камнем преткновения для многих устройств.
Достижение качественной картинки требует использования продвинутых методов визуализации, продвинутые методы визуализации требуют дополнительной видеопамяти, а устройств с неограниченной памятью человечеством на данный момент изобретено не было.
Наиболее остро вопрос расхода видеопамяти стоит на портативных устройствах: смартфонах, консолях, HMD-""устройствах.
Среди последних, например, устройства серии <<Quest>> содержат 4 и 6 гигабайт гибридной памяти, используемой и GPU, и CPU.
С суммарным разрешением 2x1832x1920, всего одно полноэкранное изображение на <<Quest 2>> занимает около 113 мегабайт.
Предположив равное использование памяти GPU и CPU, а также резервацию нескольких гигабайт под данные сцены, приложению вряд ли удастся создать больше 10 полноэкранных изображений для использования алгоритмами визуализации, 4--5 из которых, скорее всего, сразу же будут использованы G-буфером распространённой техники отложенного освещения \cite{10.1145/378456.378468}.
С другой стороны, наименее проблематичной категорией устройств считаются персональные компьютеры. Согласно обзору ПК пользователей <<Steam>> \cite{steamSurvey2023may}, медианным количеством видеопамяти являются 8 гигабайт, а медианным разрешением 1920x1080. Конечно же, такая ситуация более благоприятна для сложных методов визуализации сцен. Однако необходимо учитывать, что 1--2 гигабайта видеопамяти зачастую расходуется прочими приложениями, запущенными пользователем и операционной системой, а ожидания пользователей о качестве картинки и детализации сцены много выше чем в случае с HMD-устройствами. Более того, согласно \cite{steamSurvey2023may}, на данный момент более 20\% пользователей <<Steam>> пользуются мониторами с разрешением 4K, требующими в 4 раза большей детализации от приложения. Для оправдания всех ожиданий и требований, от разработчиков приложений требуется эффективное использование доступной видеопамяти.

Тем не менее, до относительно недавнего времени, эффективное управление видеопамятью было попросту невозможно в силу технических ограничений, рассмотрению которых посвящён следующий раздел.
