\documentclass[12pt, draft]{extarticle}

\usepackage{fontspec}
\usepackage{polyglossia}
\let\lang\relax\let\endlang\relax % avoid conflict with complexity.sty

\setdefaultlanguage{russian}
\setotherlanguage{english}


\defaultfontfeatures{Ligatures=TeX}

\setmainfont{Times New Roman}
\setsansfont{Arial}
\setmonofont[Contextuals=Alternate]{Fira Code}

\newfontfamily\cyrillicfont{Times New Roman}
\newfontfamily\cyrillicfontsf{Arial}
\newfontfamily\cyrillicfonttt[Contextuals=Alternate]{Fira Code}

\newfontfamily\englishfont{Times New Roman}
\newfontfamily\englishfontsf{Arial}
\newfontfamily\englishfonttt[Contextuals=Alternate]{Fira Code}

\usepackage{microtype} % nicer layouts with microtypography tricks

\usepackage[a4paper, lmargin=30mm, rmargin=15mm, tmargin=20mm, bmargin=20mm]{geometry}
\linespread{1.5}
\usepackage[skip = 1.25cm]{parskip}
\newcommand{\mainparskip}{1.25cm}


\usepackage[outputdir=build]{minted}
\usemintedstyle{vs}
\setminted{fontsize=\footnotesize}

\newcommand{\inlcpp}[1]{\mintinline{c++}{#1}}
\newcommand{\mintlbl}[1]{\phantomsection\label{#1}}

\usepackage[backend=biber,
  bibencoding=utf8,
  sorting=none,
  style=gost-numeric,
  language=autobib,
  autolang=other,
  clearlang=true,
  defernumbers=true,
  sortcites=true,
  doi=true,
  isbn=true,
]{biblatex}

\bibliography{bibliography}

\usepackage[dvipsnames]{xcolor}
\usepackage[hidelinks,colorlinks=true,citecolor=BurntOrange,urlcolor=Blue]{hyperref}
\renewcommand{\UrlFont}{\small\rmfamily\tt}
\renewcommand\thesection{\arabic{section}}

\usepackage{amsfonts}

\usepackage{tikz}
\usetikzlibrary{graphs, arrows.meta}
\usepackage{pgfplots}
\usepackage{pgfplotstable}
\pgfplotsset{compat=1.18}
\usepgfplotslibrary{external}
\usetikzlibrary{external}
\tikzexternalize[prefix=build/]

\usepackage{wrapfig}

\usepackage{easy-todo}

\usepackage{graphicx}
\graphicspath{ {./images/} }

% Code snippets
\usepackage[outputdir=build]{minted}
\usemintedstyle{vs}
\renewcommand{\inlcpp}[1]{\mintinline{c++}{#1}}

\usepackage{enumitem}
\setlist{parsep=0pt, topsep={-\mainparskip}}
\setlist[enumerate]{label={\arabic*)}}
\setlist[itemize]{label={---}}

\usepackage{amssymb}
\usepackage{amsmath}

\usepackage{complexity}

\usepackage{xpatch}

\usepackage{algpseudocode}
\usepackage{algorithm}
\makeatletter
\xpatchcmd{\algorithmic}{\itemsep\z@}{\itemsep=-0.2em}{}{}
\makeatother
\floatname{algorithm}{Алгоритм}


\pghyphenation{russian}{
    фрейм-граф
    рен-дер-граф
    рен-дер-инг
}


\hypersetup{final}

\begin{document}

\newgeometry{lmargin=15mm, rmargin=15mm, tmargin=20mm, bmargin=20mm}

\begin{titlepage}
    \begin{center}
    Федеральное государственное автономное образовательное учреждение\break высшего образования\par
    <<Московский физико-технический институт (государственный университет)>>\par
    Физтех-школа прикладной математики и информатики\par
    Центр обучения проектированию и разработке игр\par
    \end{center}
%
    {\bf Направление подготовки}: 09.04.01 Информатика и вычислительная техника\newline
    {\bf Направленность (профиль) подготовки}: Анализ данных и разработка информационных систем\par
%
    {
    \topskip0pt
    \vspace*{\fill}
    \begin{center}
        {\bf\LARGE Оптимизация потребления видеопамяти
        \break при помощи вычислительного графа
        \break в приложениях реального времени}\par
        (магистерская диссертация)
    \end{center}
    \vspace*{\fill}
    }
%
    \hfill
    \begin{minipage}[t]{7cm}
    {\bf Студент: \newline}
    Санду Роман Александрович\newline
    \vspace{-3mm}
    \rule{7cm}{0.15mm}
    \centerline{\small\it (подпись студента)}\newline
    {\bf Научный руководитель: \newline}
    Щербаков Александр Станиславович\newline
    \vspace{-3mm}
    \rule{7cm}{0.15mm}
    \centerline{\small\it (подпись научного руководителя)}
    \end{minipage}
    \vspace*{\fill}
    \begin{center}
        Москва 2023
    \end{center}
\end{titlepage}

\restoregeometry



\newpage
\setcounter{page}{2}

\vspace*{\fill}
\begin{abstract}
Данная работа посвящена одному из подходов к построению архитектуры приложений реального времени, называемого неформально "фреймграфом" или "рендерграфом". Подход основывается на использовании вычислительного графа как представления процесса вычисления итоговой картинки одного кадра приложения. 
\end{abstract}
\vspace*{\fill}

\newpage
\tableofcontents
\newpage

\setlength{\parskip}{\mainparskip}

\section{Введение}
С самого появления области интерактивных приложений реального времени с трёхмерной графикой, главной задачей является максимизация качества картинки без потери производительности.
Так, основным бизнес-""требованием от индустрий кино, игр, виртуальных симуляторов, промышленной визуализации, является реализм итогового изображения.
С целью удовлетворения этих требований было проведено множество академических исследований различных методов визуализации, разработаны новые алгоритмы и техники отрисовки сцен.
Параллельно с этим не стояла на месте и область потребительской и промышленной техники.
В настоящий день не редка ситуация, когда одно и то же приложение должно запускаться на целом ряде различных устройств: персональных компьютерах абсолютно разной комплектации, стационарных и портативных игровых консолях, мобильных устройствах различных производителей и даже HMD-устройствах виртуальной и дополненной реальности\footnote{Устройства, закрепляемые на голове пользователя, от англ. head mounted device.}.

Вопрос потребления видеопамяти такими приложениями является камнем преткновения для многих устройств.
Достижение качественной картинки требует использования продвинутых методов визуализации, продвинутые методы визуализации требуют дополнительной видеопамяти, а устройств с неограниченной памятью человечеством на данный момент изобретено не было.
Наиболее остро вопрос расхода видеопамяти стоит на портативных устройствах: смартфонах, консолях, HMD-""устройствах.
Среди последних, например, устройства серии <<Quest>> содержат 4 и 6 гигабайт гибридной памяти, используемой и GPU, и CPU.
С суммарным разрешением 2x1832x1920, всего одно полноэкранное изображение на <<Quest 2>> занимает около 113 мегабайт.
Предположив равное использование памяти GPU и CPU, а также резервацию нескольких гигабайт под данные сцены, приложению вряд ли удастся создать больше 10 полноэкранных изображений для использования алгоритмами визуализации, 4--5 из которых скорее всего сразу же будут использованы G-буфером распространённой техники отложенного освещения.
С другой стороны, наименее проблематичной категорией устройств считаются персональные компьютеры. Согласно обзору ПК пользователей <<Steam>> \cite{steamSurvey2023may}, медианным количеством видеопамяти являются 8 гигабайт, а медианным разрешением 1920x1080. Конечно же, такая ситуация более благоприятна для сложных методов визуализации сцен. Однако необходимо учитывать, что 1--2 гигабайта видеопамяти зачастую расходуется прочими приложениями, запущенными пользователем и операционной системой, а ожидания пользователей о качестве картинки и детализации сцены много выше чем в случае с HMD-устройствами. Более того, согласно \cite{steamSurvey2023may}, на данный момент более 20\% пользователей <<Steam>> пользуются мониторами в 4К разрешении, требующими в 4 раза большей детализации от приложения. Для оправдания всех ожиданий и требований, от разработчиков приложений требуется эффективное использование доступной видеопамяти.

Тем не менее, до относительно недавнего времени, эффективное управление видеопамятью было попросту невозможно в силу технических ограничений, рассмотрению которых посвящён следующий раздел.

\section{Обзор существующих работ}
\subsection{Имплементации}

\subsubsection*{Frostbite}
EA выступление\cite{FrostbiteGdcTalk}

\subsubsection*{Halcyon}
EA выступление\cite{HalcyonRapidInnovationTalk}
идеален во всём, но пока только R\&D

\subsubsection*{Unity}
документация\cite{UnityRenderGraph}
закрытая, но вроде хорошая

\subsubsection*{Unreal Engine}
документация\cite{UERenderDependencyGraph}

\subsubsection*{Anvil}
Ubisoft выступление\cite{DX12CaseStudies}
есть алиасинг, есть автобарьеры (сплит), умеет в несколько очередей сабмита

\subsubsection*{Granite}
блог\cite{GraniteBlogPost}

\subsubsection*{Прочие}
Неинтересные:
https://github.com/azhirnov/FrameGraph -- нет алиасинга, очень много ООП, намертво привязан к вулкану, вершины не реордерятся, содержимое вершин -- фиксированные таски, а не произвольный код, нет хистори ресурсов, есть барьеры, ВРОДЕ БЫ нет алиасинга
https://github.com/skaarj1989/FrameGraph -- нет алиасинга, нет хистори ресурсов, нет барьеров, кросс-АПИ, прикольный интерфейс на C++, видимо заброшен
https://github.com/Raikiri/LegitEngine -- ОТЕЧЕСТВЕННОЕ!!!


\newpage
\printbibliography
\end{document}

