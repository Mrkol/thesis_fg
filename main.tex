\documentclass[14pt]{extarticle}

\usepackage{fontspec}
\usepackage{polyglossia}


\setdefaultlanguage{russian}
\setotherlanguage{english}


\defaultfontfeatures{Ligatures=TeX}

\setmainfont{Times New Roman}
\setsansfont{Arial}
\setmonofont[Contextuals=Alternate]{Fira Code}

\newfontfamily\cyrillicfont{Times New Roman}
\newfontfamily\cyrillicfontsf{Arial}
\newfontfamily\cyrillicfonttt[Contextuals=Alternate]{Fira Code}

\newfontfamily\englishfont{Times New Roman}
\newfontfamily\englishfontsf{Arial}
\newfontfamily\englishfonttt[Contextuals=Alternate]{Fira Code}

\usepackage[a4paper, lmargin=30mm, rmargin=15mm, tmargin=20mm, bmargin=20mm]{geometry}
\linespread{1.5}
\usepackage{parskip}
\newcommand{\mainparskip}{1.25cm}


\usepackage[outputdir=build]{minted}
\usemintedstyle{vs}
\setminted{fontsize=\footnotesize}

\newcommand{\inlcpp}[1]{\mintinline{c++}{#1}}

\usepackage[backend=biber,
  bibencoding=utf8,
  sorting=none,
  style=gost-numeric,
  language=autobib,
  autolang=other,
  clearlang=true,
  defernumbers=true,
  sortcites=true,
  doi=true,
  isbn=true,
]{biblatex}

\bibliography{bibliography}

\usepackage[dvipsnames]{xcolor}
\usepackage[hidelinks,colorlinks=true,citecolor=BurntOrange,urlcolor=Blue]{hyperref}
\renewcommand{\UrlFont}{\small\rmfamily\tt}
\renewcommand\thesection{\arabic{section}}

\usepackage{amsfonts}

\usepackage{tikz}
\usepackage{pgfplots}
\pgfplotsset{compat=1.18}
\usepgfplotslibrary{external}
\tikzexternalize

\usepackage{wrapfig}

\usepackage{easy-todo}

\usepackage{graphicx}
\graphicspath{ {./images/} }

% Code snippets
\usepackage[outputdir=build]{minted}
\usemintedstyle{vs}
\renewcommand{\inlcpp}[1]{\mintinline{c++}{#1}}

\usepackage{enumitem}
\setlist{parsep=0pt, topsep={-\mainparskip}}
\setlist[enumerate]{label={\arabic*)}}
\setlist[itemize]{label={---}}

\usepackage{amssymb}


\pghyphenation{russian}{
    фрейм-граф
    рен-дер-граф
    рен-дер-инг
}


\hypersetup{final}

\begin{document}

% \newgeometry{lmargin=15mm, rmargin=15mm, tmargin=20mm, bmargin=20mm}

\begin{titlepage}
    \begin{center}
    Федеральное государственное автономное образовательное учреждение\break высшего образования\par
    <<Московский физико-технический институт (государственный университет)>>\par
    Физтех-школа прикладной математики и информатики\par
    Центр обучения проектированию и разработке игр\par
    \end{center}
%
    {\bf Направление подготовки}: 09.04.01 Информатика и вычислительная техника\newline
    {\bf Направленность (профиль) подготовки}: Анализ данных и разработка информационных систем\par
%
    {
    \topskip0pt
    \vspace*{\fill}
    \begin{center}
        {\bf\LARGE Архитектура рендеринга реального времени 
        \break через вычислительный граф}\par
        (магистерская диссертация)
    \end{center}
    \vspace*{\fill}
    }
%
    \hfill
    \begin{minipage}[t]{7cm}
    {\bf Студент: \newline}
    Санду Роман Александрович\newline
    \vspace{-3mm}
    \rule{7cm}{0.15mm}
    \centerline{\small\it (подпись студента)}\newline
    {\bf Научный руководитель: \newline}
    Щербаков Александр Станиславович\newline
    \vspace{-3mm}
    \rule{7cm}{0.15mm}
    \centerline{\small\it (подпись научного руководителя)}
    \end{minipage}
    \vspace*{\fill}
    \begin{center}
        Москва 2023
    \end{center}
\end{titlepage}
    
\restoregeometry



\newpage
\setcounter{page}{2}

\vspace*{\fill}
\begin{abstract}
Данная работа посвящена вопросу экономии видеопамяти, потребляемой в процессе отрисовки кадра графического приложения реального времени промежуточными структурами данных алгоритмов визуализации сцен.
Широко распространившийся подход структурирования вычисления кадра в виде графа позволяет более качественно управлять аллокациями видеопамяти, а также автоматизировать различные аспекты разработки современных трёхмерных приложений.
В рамках работы предложен комплексное решение задания кадровых графов, вопрос управления памятью временных ресурсов сведён к задаче дискретной оптимизации, а также предложена качественная, эффективная схема решения этой задачи.
Преимущество предложенной схемы над существующими заключается в переиспользовании памяти, используемой ресурсами темпоральных алгоритмов, широко распространённых среди современных методов визуализации трёхмерных сцен.
\end{abstract}
\vspace*{\fill}

\newpage
\tableofcontents
\newpage

\section{Введение}
С самого появления области интерактивных приложений реального времени с трёхмерной графикой, главной задачей является максимизация качества картинки без потери производительности.
Так, основным бизнес"=требованием от индустрий кино, игр, виртуальных симуляторов, промышленной визуализации, является реализм итогового изображения.
С целью удовлетворения этих требований было проведено множество академических исследований различных методов визуализации, разработаны новые алгоритмы и техники отрисовки сцен.
Параллельно с этим не стояла на месте и область потребительской и промышленной техники.
В настоящий день не редка ситуация, когда одно и то же приложение должно запускаться на целом ряде различных устройств: персональных компьютерах абсолютно разной комплектации, стационарных и портативных игровых консолях, мобильных устройствах различных производителей и даже HMD-устройствах виртуальной и дополненной реальности\footnote{Устройства, закрепляемые на голове пользователя, от англ. head mounted device.}.

Вопрос потребления видеопамяти такими приложениями является камнем преткновения для многих устройств.
Достижение качественной картинки требует использования продвинутых методов визуализации, продвинутые методы визуализации требуют дополнительной видеопамяти, а устройств с неограниченной памятью человечеством на данный момент изобретено не было.
Наиболее остро вопрос расхода видеопамяти стоит на портативных устройствах: смартфонах, консолях, HMD-""устройствах.
Среди последних, например, устройства серии <<Quest>> содержат 4 и 6 гигабайт гибридной памяти, используемой и GPU, и CPU.
С суммарным разрешением 2x1832x1920, всего одно полноэкранное изображение на <<Quest 2>> занимает около 113 мегабайт.
Предположив равное использование памяти GPU и CPU, а также резервацию нескольких гигабайт под данные сцены, приложению вряд ли удастся создать больше 10 полноэкранных изображений для использования алгоритмами визуализации, 4--5 из которых, скорее всего, сразу же будут использованы G-буфером распространённой техники отложенного освещения \cite{10.1145/54852.378468}.
С другой стороны, наименее проблематичной категорией устройств считаются персональные компьютеры. Согласно обзору ПК пользователей <<Steam>> \cite{steamSurvey2023may}, медианным количеством видеопамяти являются 8 гигабайт, а медианным разрешением 1920x1080. Конечно же, такая ситуация более благоприятна для сложных методов визуализации сцен. Однако необходимо учитывать, что 1--2 гигабайта видеопамяти зачастую расходуется прочими приложениями, запущенными пользователем и операционной системой, а ожидания пользователей о качестве картинки и детализации сцены много выше чем в случае с HMD-устройствами. Более того, согласно \cite{steamSurvey2023may}, на данный момент более 20\% пользователей <<Steam>> пользуются мониторами в 4К разрешении, требующими в 4 раза большей детализации от приложения. Для оправдания всех ожиданий и требований, от разработчиков приложений требуется эффективное использование доступной видеопамяти.

Тем не менее, до относительно недавнего времени, эффективное управление видеопамятью было попросту невозможно в силу технических ограничений, рассмотрению которых посвящён следующий раздел.

\section{Введение в предметную область}
Около 10 лет назад история развития компьютерной графики реального времени потерпела кардинальный поворот с выходом графических API нового поколения, DirectX12, Vulkan и Metal.
Их предшественники, OpenGL и DirectX ранних версий, основывались на идее так называемого <<толстого>> драйвера.
Дизайн этих API старался максимально скрыть принципы работы видеокарт, предоставляя пользователям простой, но достаточно ограниченный инструмент для разработки графических приложений реального времени.
По мере развития индустрии компьютерной графики, разработчики приложений всё чаще сталкивались с ограничениями этого дизайна, а простота дизайна всё больше жертвовалась в пользу поддержки новых возможностей различных типов видеокарт.
Так, например, печально известна неоправданная сложность разработки трёхмерных приложений для мобильных платформ с использованием <<OpenGL for Embedded Systems>>, вызванная спецификой графических ускорителей, используемых на этих платформах.

Главной целью дизайна графических API нового поколения было избавление от накладных расходов сложных абстракций, вследствие чего произошёл отказ от <<толстых>> драйверов в пользу раскрытия всё большего числа внутренних деталей работы GPU.
Доступ к низкоуровневым механизмам графических ускорителей позволил приложениям лучше адаптироваться под конкретные устройства, рациональнее использовать доступные ресурсы, и, как следствие, достигать большего качества и производительности.
Однако сложность новых API требует от разработчика сильно более структурированного подхода к разработке приложений, дизайна собственных абстракций, собственных алгоритмов и систем управления различными ресурсами.
В следующем подразделе более детально освещены различные аспекты перехода к новым API, а также проблемы, возникающие при разработке приложений с их использованием.

\subsection{Аспекты разработки графических приложений реального времени в старых и новых API}
\subsubsection{Управление памятью транзиентных ресурсов}
Главный интерес в рамках данной работы составляют кардинальные изменения в работе с видеопамятью.
В процессе вычисления картинки одного кадра любое нетривиальное приложение использует \textit{транзиентные ресурсы} "--- промежуточные хранилища данных, содержимое которых не требуется после окончания вычисления кадра, либо требуется лишь в процессе вычисления следующего кадра.
Как правило, подобные ресурсы являются картинками с разрешением кратным разрешению монитора пользователя.
Из этого следует, что при переходе от 1080p мониторов к 4K мониторам потребление памяти транзиентными ресурсами возрастает в 4 раза, что обуславливает нужду в эффективном её переиспользовании.
Старые графические API полностью скрывали управление памятью GPU от пользователя, предоставляя лишь функции создания и удаления конкретных ресурсов.
За годы существования этой абстракции образовалось 3 основных подхода к эффективному управлению памятью транзиентных ресурсов.

Самым простым подходом является выделение и освобождение транзиентных ресурсов по ходу их нужды при помощи соответствующих вызовов графического API.
Этот подход фактически идентичен выделению памяти в различных языках программирования: драйвер операционной системы содержит аллокатор, на который пользователь перекладывает обязанность управления памятью и другими ресурсами GPU, аналогично аллокациям на куче в языке C.
Системный аллокатор переиспользует освободившуюся память, тем самым достигая низкого её потребления.
Однако такой подход не масштабируется на более сложные приложения.
Во-первых, известны нижние оценки на качество работы интерактивных аллокаторов (см. \cite{robson1971estimate}), на практике выражающиеся как фрагментация кучи.
Во-вторых, как правило, и создание, и освобождение ресурсов являются весьма дорогими операциями вследствие деталей реализации драйверов.

В некотором смысле противоположным подходом служит отказ от переиспользования памяти.
Все транзиентные ресурсы создаются заранее и не удаляются в ходе работы приложения.
Очевидно, что с повышением сложности приложения такой подход перестаёт быть применимым, что иногда приводит к попыткам вручную переиспользовать некоторые выделенные объекты.
Это, в свою очередь, приводит к чрезвычайно сложному для понимания коду, усложняя и замедляя работу над самим приложением.

Наконец, наиболее практичным подходом является техника \textit{пулирования} ресурсов.
Вся программа работает с объектом называемым \textit{пулом}, отвечающим за выделение и освобождение ресурсов.
Пул использует механизмы драйвера для выделения новых ресурсов, но вместо освобождения ресурсов в драйвер хранит список неиспользуемых ресурсов конкретного типа (в понятие тип, как правило, входит разрешение для текстур и размер для буферов соответственно, а также все флаги свойств ресурса).
Последующие запросы на выделение ресурсов обслуживаются в первую очередь из списка неиспользуемых, и только исчерпав его выделяются новые ресурсы посредством драйвера операционной системы.
Данный подход был оптимален до появления современных низкоуровневых графических API.
Однако в настоящее время хорошо заметен его главный недостаток: память не переиспользуется между ресурсами разных типов.

В современных же графических API предоставляется прямой, неограниченный доступ к памяти GPU.
Приложение может выделять \textit{кучи}, последовательности страниц виртуальной видеопамяти, и затем создавать ресурсы на конкретных адресах в рамках конкретной кучи.
Стоит отметить, что пересечение используемых разными ресурсами регионов кучи не запрещается, хоть поведение при одновременном использовании таких ресурсов не определено.
Фактически, это нововведение перекладывает ответственность по написанию аллокатора ресурсов с разработчиков драйвера на разработчиков приложения, что можно сравнить с разработкой на языке C, используя лишь системные вызовы POSIX \inlcpp{mmap} и \inlcpp{munmap} вместо аллокатора библиотеки glibc.
С одной стороны, это сильно усложняет разработку простых приложений.
Вследствие этого компания AMD открыла исходный код аллокатора из своих драйверов \cite{VMA}, к использованию которого нередко прибегают даже в промышленных приложениях, что, конечно же, возвращает статус-кво старых графических API.
С другой стороны, это даёт возможность разработчикам более эффективно распоряжаться видеопамятью в различных подсистемах приложения, в частности, позволяя построить в некотором смысле оптимальное расписание аллокации транзиентных ресурсов, чему в первую очередь и посвящена данная работа.

\subsubsection{Отправка команд GPU}
Работа с любыми внешними по отношению к центральному процессору компонентами компьютера по своей природе асинхронна.
Передача данных по проводам занимает время, как и их обработка на внешнем устройстве.
Заставлять ядра центрального процессора простаивать в ожидании отклика от внешнего устройства "--- непозволительная растрата ресурсов.
Не исключение и графические ускорители.
Низкоуровневым инструментом для общения GPU и операционной системы служат \textit{списки команд}, состоящие из команд отрисовки, запуска вычислений, синхронизации, копирования данных, и прочих.
В случае если центральному процессору необходимо дождаться результата каких-либо вычислений на GPU, ожидание необходимо делать вручную, используя аппаратные сигналы о прогрессе от видеоускорителя.

В старых API асинхронная природа вычислений на GPU скрывалась за абстракцией \textit{мгновенного режима} (от англ. immediate mode).
Драйвер создавал видимость мгновенного выполнения всех команд GPU, представлявших собой функции.
Примитивы синхронизации в рамках GPU, а также между GPU и CPU, вставлялись автоматически, что не редко приводило к непредсказуемой производительности кода.
Более того, производительность крупных приложений могла упираться в скорость записи командных буферов внутри драйвера.
Естественным способом решения этой проблемы была бы параллельная их запись, но машина состояний внутри драйверов старых графических API была фундаментально однопоточной структурой.
Далее, в какой-то момент времени графические ускорители начали поддерживать параллельную обработку нескольких очередей команд.
Эта возможность может давать прирост производительности при слабой загрузке вычислительных модулей GPU исполняемыми командами.
Дизайн старых API не был рассчитан на поддержку таких возможностей аппаратуры, что сильно усложняло работу с этим функционалом.

С приходом новых графических API фактически все проблемы с отправкой команд были решены, а точнее ответственность по их решению была переложена на разработчиков приложений.
Новые API предоставляют пользователю прямой доступ к спискам команд, примитивам синхронизации и очередям исполнения команд.
Однако как и в случае с прямым доступом к памяти, работа напрямую с предоставляемыми графическим API абстракциями не является практичной, необходима разработка собственных абстракций для удобной записи команд.

\subsubsection{Управление кешами GPU и состоянием ресурсов}
Характерным отличием графических ускорителей от центральных процессоров является отсутствие гарантий когерентности кешей, и вследствие чего необходимость в ручную делать их инвалидацию и сброс.
Однако старые графические API скрывали эту особенность аппаратуры, автоматически отслеживая состояние ресурса и вставляя соответствующие команды синхронизации в список команд.
Недостаток такого подхода заключается в отсутствие невозможности предсказывать последующие команды пользователя на уровне драйвера, что не редко приводит к исполнению команд синхронизации в неоптимальный момент времени.
Усугубляет ситуацию тот факт, что некоторые GPU используют различные оптимизации формата хранения ресурсов при их использовании конкретным образом, и переход между разными оптимизированными состояниями приводит к простою вычислительных ресурсов GPU.

В новых же графических API ответственность за отслеживание состояния ресурсов и кешей переложена на пользователя посредством абстракции \textit{барьеров}.
Барьеры служат главным примитивом синхронизации в рамках GPU, управления кешами и состояниями ресурсов.
Однако расстановка барьеров в корректных и оптимальных местах порой оказывается далеко не тривиальной задачей, требующей от пользователя глобального понимания работы всей системы.
Это делает модуляризацию приложения невозможным без введения специальных механизмов для расстановки барьеров.

\subsection{Кадровые графы}
Сложно отследить появление понятия вычислительного графа, однако первым приложением этой техники в контексте графических приложений реального времени считается игровой движок Frostbite компании EA, о чём было объявлено в 2017 году на конференции Game Developers Conference \cite{FrostbiteGdcTalk}.
С тех пор большая часть коммерческих движков перешла на архитектуру, основанную на кадровом графе, о чём подробнее в следующем разделе.
Причина этого перехода следующая: наличие кадрового графа в виде данных позволяет решить все вышеописанные проблемы, появившиеся с приходом новых графических API, при этом сделать это оптимальнее чем позволяли встроенные в драйвер механизмы старых API.

Уточним используемую в дальнейшем терминологию.
\textit{Кадровым графом} назовём конкретный набор вершин, рёбер и других данных, определяемый настройками и спецификой приложения, и задающий глобальную структуру процесса вычисления одного кадра.
\textit{Рантаймом} кадровых графов будем называть программное решение, принимающее на вход конкретный кадровый граф и позволяющее его \textit{компилировать}, запускать и, возможно, редактировать.
Процесс компиляции кадрового графа "--- некоторый набор действий и вычислений, которые необходимо совершить перед запуском кадрового графа.
Конкретный набор данных, задающий кадровый граф, определяется дизайном конкретного рантайма, как и специфика процессов компиляции, запуска, а также дополнительный функционал, предоставляемый пользователям рантайма, а именно разработчикам алгоритмов визуализации сцены.
Пространство дизайна рантайма кадровых графов весьма широко, но независимо от конкретных решений, рантайм будет принимать на вход набор вершин, каждая из которых содержит \textit{функцию запуска}, инициирующую некоторые вычисления на GPU, и список используемых функцией запуска ресурсов с дополнительной информацией о способе использования.
Рёбра же графа, как правило, не задаются явно, а создаются автоматически, например, между вершиной создающей некоторый ресурс и читающей этот ресурс вершиной.
Абстрагируясь от конкретики, опишем как именно архитектура, основанная на кадровом графе, позволяет решить описанные выше проблемы.

\subsubsection{Управление памятью в кадровых графах}
Во-первых, имея в виде данных глобальную информацию об использовании ресурсов на протяжении всего кадра, вопрос управления памятью GPU сводиться к неинтерактивной вариации хорошо известной в литературе задачи динамической аллокации памяти \cite[с. 226]{10.5555/574848}.
В отсутствие же глобальной информации управление памятью является интерактивной вариацией этой задачи.
Как уже было упомянуто ранее, известно, что алгоритмы для интерактивной вариации этой задачи работают качественно хуже, чем для статической (см. \cite{robson1971estimate}), и более того, создание ресурсов GPU "--- достаточно дорогая операция, а качественные алгоритмы решения интерактивной динамической аллокации памяти занимают значительное время.
Эти факторы делают решение неинтерактивной вариации этой задачи единожды в момент компиляции кадрового графа привлекательной идеей, и наоборот, скорее всего любое решение оптимально управляющее памятью транзиентных ресурсов в том или ином виде будет сводится к кадровому графу.

\subsubsection{Кеши и состояние ресурсов в кадровом графе}
Во-вторых, если рантайм требует от пользователей указывать конкретный способ использования ресурсов внутри вершины, то в процессе запуска кадрового графа становится возможным автоматически расставлять барьеры.
Если ресурс был создан и заполнен данными посредством рендеринга вершиной $A$, а затем после запуска некоторого количества других вершин был семплирован вершиной $B$, то рантайм обязан поставить барьер переводящий ресурс из состояния пригодного для рендеринга в состояние пригодное для семплирования.
Однако, из-за наличия нескольких промежуточных вершин, у рантайма есть выбор, в какой конкретно момент поставить барьер.
Этот выбор может влиять на производительность из-за конвейеризации вычислений на современных графических ускорителях, а значит можно сформулировать задачу дискретной оптимизации расстановки барьеров с целью минимизации простоев вычислительных модулей GPU.
Впрочем, постановка и решение такой задачи выходит за рамки данной работы.

\subsubsection{Кадровые графы и GPU архитектуры TBDR}
В-третьих, важной особенностью современных графических API, не упомянутой ранее, является поддержка специфичной для портативных и мобильных устройств архитектуры графических ускорителей, называемой <<Tile Based Deferred Renderer>> (TBDR).
Не вдаваясь в детали принципов работы TBDR, старые графические API были абсолютно не рассчитаны на подобные возможности апаратуры, что приводило к непредсказуемому влиянию изменений в коде на производительность, а также делало некоторые техники визуализации вроде отложенного освещения неприменимыми на подобных устройствах.
В новых же API был предоставлен почти полный контроль над механизмами работы TBDR, что с одной стороны решило эти проблемы, но с другой стороны сильно усложнило процесс разработки из-за введения понятия \textit{рендер"=пасса} \cite[раздел~8]{VulkanSpec}.
Введение слоя абстракции над графическим API в виде кадрового графа позволяет одновременно и упростить интерфейс предоставляемый пользователю для работы с TBDR и рендер"=пассами, и сохранить предсказуемость производительности, отчасти путём предоставления инструментов визуализации структуры графа.

\subsubsection{Отправка команд GPU из кадрового графа}
В-четвёртых, как уже было упомянуто выше, современные GPU часто предоставляют разработчикам несколько очередей для отправки команд, что позволяет лучше насытить вычислительные модули в ситуации когда выполняемые команды не требуют стопроцентного использования всех возможностей ускорителя.
Рантайм кадровых графов может помочь автоматизировать этот процесс, заранее выбирая стратегию планирования имеющихся вершин на имеющиеся очереди команд.
Далее, даже в рамках одной очереди команд не редко имеет смысл чередовать команды от независимых ветвей графа с целью лучше насытить вычислительные ресурсы GPU полезной работой.
Более того, само исполнение вершин кадрового графа для получения списков команд рантайм может производить параллельно для независимых ветвей графа, что может понизить время затрачиваемое на кадр на центральном процессоре.
В данной работе, однако, более глубоко данный вопрос не освещается.

\subsubsection{Архитектурные аспекты кадровых графов}
Наконец, вероятно самым важным фактором в распространении кадровых графов послужили архитектурные преимущества этого подхода.
Как было сказано ранее, рёбра графа обычно задаются пользователями рантайма неявно, посредством ресурсных зависимостей.
Это позволяет добиться низкой связности различных подсистем приложения: модулям не требуется знать о прочей системе ничего, кроме названий ресурсов и публичного API рантайма кадровых графов.
Более того, рантайм может автоматически обнаруживать некорректные конфигурации приложения, в которых тем или иным модулям не хватает необходимых ресурсных зависимостей, отключая соответствующие модули автоматически.

Конечно же, далеко не все имплементации рантаймов кадровых графов содержат решения всех перечисленных проблем, и не всегда в описанном выше виде.
Некоторые же имплементации содержат уникальный для них функционал, решающий специфичные для конкретной области проблемы.
Обзор существующих комплексных решений содержится в следующем разделе работы.

\section{Цели и задачи}
Цель данной работы "--- разработка схемы оптимизации потребления памяти транзиентными ресурсами, в том числе используемыми в темпоральных алгоритмах. Исходя из этого, был поставлен следующий ряд задач:
\begin{itemize}
    \item обзор существующих решений,
    \item формулировка алгоритмической задачи управления видеопамятью,
    \item разработка качественного метода решения задачи управления видеопамятью,
    \item проведение анализа потребления памяти и производительности.
\end{itemize}
Как уже было сказано выше, оптимальное управление видеопамятью требует глобальных знаний о процессе вычисления кадра, что в свою очередь фактически эквивалентно использованию кадрового графа.
В силу этого удовлетворительное с практической точки зрения решение последних трёх задач также требует решения следующих инженерных подзадач:
\begin{itemize}
    \item разработка рантайма кадровых графов в рамках движка Dagor, обладающего достаточным функционалом для интеграции в существующие проекты;
    \item разработка эргономичного пользовательского API для рантайма;
    \item непосредственная интеграция разработанного решения в проект, основанный на движке Dagor, <<Enlisted>>.
\end{itemize}

\section{Обзор существующих работ}
\subsection{Рантаймы кадровых графов}
Переходя к обзору различных имплементаций рантайма кадровых графов, стоит отметить, что далеко не все компании готовы рассказывать об используемых в их разработках технологиях. Список, приведённый ниже, включает лишь все коммерческие разработки, о которых достоверно известно из открытых источников информации, что архитектура их основана на вычислительном графе.
\subsubsection*{Frostbite}
Первыми идею организации архитектуры рендеринга в приложениях реального времени через вычислительные графы предложили разработчики движка Frostbite в 2017 году \cite{FrostbiteGdcTalk}.
\textit{Кадровый граф} позволил им сделать ядро модуля рендеринга расширяемым, упростил работу с асинхронным вычислениями общего назначения на GPU, автоматизировал работу со специализированными видами оперативной видеопамяти на игровых консолях, а также сэкономил большое количество обычной видеопамяти.
В силу проприетарности движка неизвестно, насколько широкий класс сценариев использования ресурсов она поддерживает.
В качестве схемы аллокации ресурсов же был взят обычный интерактивный аллокатор, располагающий в заранее выделенном крупном участке памяти ресурсы по мере необходимости.
Автоматическая расстановка барьеров на 2017 год не поддерживалась.

\subsubsection*{Halcyon}
Далее, в 2019 году, компания EA представила \cite{HalcyonRapidInnovationTalk} новый экспериментальный движок Halcyon, обобщающий идею кадрового графа до \textit{графа рендеринга}.
Как следует из названия, это обобщение позволяет организовывать в виде графа вычислений не только процесс рендеринга самого кадра, но и рендеринг различных вспомогательных изображений, например, кубических карт для некоторых техник глобального освещения, изображений импосторов \cite{10.1145/199404.199420}, или различных иконок в приложении.
Более того, граф рендеринга может состоять из нескольких подграфов, запускающихся с разной частотой, например, подграф, вычисляющий тени от солнца, может запускаться только при значительном изменении положения солнца на небе.
В отличие от Frostbite, граф Halcyon поддерживает автоматическую расстановку барьеров.
Также на выступлении отмечается, что изначально граф составлялся явной композицией вершин и подграфов, но в итоге разработчики пришли к дизайну с автоматической композицией вершин на основе глобально видимых имён ресурсов.
Про алгоритм аллокации ресурсов и поддержку переживающих границу кадра ресурсов публично доступной информации нет.

Наконец, граф рендеринга Halcyon способен в автоматическом режиме масштабироваться на несколько аппаратных графических ускорителей, и даже на несколько компьютеров.
Поддержка такого функционала весьма сложна и говорит об экспериментальности этой разработки, так как на практике такая масштабируемость редко применима.

\subsubsection*{Unreal Engine}
Начиная с версии 4.22 в Unreal Engine начал переходить на рендеринг через систему <<Render Dependency Graph>> \cite{UERenderDependencyGraph}, представляющую собой граф рендеринга.
Однако даже в пятой версии движка эта система использует жадную интерактивную стратегию аллокации ресурсов, хоть и поддерживает большое количество важного функционала: автоматизацию асинхронных вычислений на GPU, расстановку барьеров и параллелизацию исполнения графа.

\subsubsection*{Unity}
Разработчики движка Unity, следуя общему направлению индустрии, в 2018 году перевели архитектуру рендеринга на подход вычислительных графов \cite{UnityRenderingArchitectureTalk}.
Однако среди прочих проприетарных движков про граф рендеринга Unity известно, пожалуй, меньше всего.
Отметить стоит лишь наличие интеграции между нативным и скриптовым кодом рендеринга, позволяющей сократить время итерации при разработке приложений.

\subsubsection*{Anvil}
Движок Anvil компании Ubisoft с переходом на DirectX12 тоже начал выделять подсистему зависимостей ресурсов, хоть и не называя её графом кадра \cite{DX12CaseStudies, AnvilDx12LessonsLearned}.
Согласно выступлениям на GDC, эта система поддерживает многий функционал уже упомянутых: переиспользует память ресурсов, автоматически расставляет барьеры, автоматизирует асинхронные вычисления.
Но, как и в случае Unity, детали устройства этой разработки отсутствуют в публичном доступе.

\subsubsection*{Render Pipeline Shaders}
Выпущенная в открытый доступ \cite{RPSgithub} в декабре 2022 года библиотека Render Pipeline Shaders компании AMD в своём составе имеет комплексное решение для построения кадровых графов \cite{RPSpost}.
Эта библиотека почти полностью скрывает от пользователя управление транзиентными ресурсами посредством предметно"=ориентированного языка <<RPSL>>, автоматически переиспользуя память, расставляя барьеры и клонируя объявленные единожды вершины графа. Разработанное AMD решение также поддерживает стандартный функционал кадровых графов: экономия памяти посредством интерактивного алгоритма аллокации, расстановку барьеров и асинхронные вычисления. Наибольшим его преимуществом, вероятно, является исходный открытый код. Недостатком этой разработки является отсутствие переиспользования памяти \textit{темпоральных} ресурсов. В терминах данной работы, любые ресурсы, чья история была запрошена, исключаются из алгоритма аллокации и хранятся в отдельных аллокациях.

\subsubsection*{Granite}
Наконец, стоит упомянуть о существовании многих любительских проектов по написанию обобщённой библиотеки кадровых графов. Большинство из них находятся на стадии зарождения и не заслуживают подробного рассмотрения.
Исключением является проект Granite \cite{GraniteBlogPost}, в рамках которого разработан кадровый граф, адаптированный для использования на мобильных устройствах посредством API Vulkan.
Кадровый граф Granite автоматически расставляет примитивы синхронизации, группирует вершины в рендер"=пассы \cite[раздел~8]{VulkanSpec}, оптимизирует порядок исполнения вершин с точки зрения минимизации накладных расходов на синхронизацию, а также переиспользует память, хоть и при помощи жадного алгоритма аллокации.

\subsection{Аллокация ресурсов}
Задача поиска расписания аллокации ресурсов в графе кадра в своей простейшей формулировке является классической сильно NP-сложной \cite{DSAnpcomplete} задачей \textit{динамической аллокации памяти} (dynamic storage allocation, DSA \cite[с. 226]{10.5555/574848}).
У этой задачи существует две интерпретации, интерактивная и неинтерактивная.
Первая подразумевает обработку разнесённых во времени запросов на аллокацию и деаллокацию ресурсов, иначе говоря, решения об адресах ресурсов в памяти необходимо принимать в порядке времён появления ресурсов.
Этот частный случай часто встречается в операционных системах и рантаймах языков программирования.
Вторая же интерпретация подразумевает наличие заранее известных времён жизни всех ресурсов.
В рамках данной работы нас интересует именно неинтерактивная интерпретация, поэтому, в отсутствие уточнения, под задачей о динамической аллокации памяти мы будем подразумевать именно её.

Одним из первых полиномиальных алгоритмов, предложенных для решения задачи DSA, является алгоритм First-Fit \cite{chrobak_packing_1988}, работающий, как было вскоре доказано Кирстедом, с константной ошибкой не более чем в 80 раз \cite{kierstead_linearity_1988}.
Тремя годами позже Кирстед представил алгоритм с ошибкой не более чем в 6 раз \cite{kierstead_polynomial_1991}.
Эти и другие ранние работы объединяет общий подход сведения DSA к частному случаю с единичным размером всех ресурсов, эквивалентному покраске интервального графа, и последующим применением интерактивного алгоритма покраски.
Через несколько лет Йордан Гергов, отказавшись от сведения к интервальным графам, смог понизить верхнюю оценку минимальной возможной ошибки до 5 \cite{gergov_approximation_1996}, а впоследствии и до 3 \cite{gergov_algorithms_1999}.
Наконец, наилучший на данный момент результат был получен исследователями из AT\&T Labs совместно с коллегой из Ecole Polytechnique \cite{buchsbaum_opt_2003}: полиномиальный алгоритм, для любого заранее выбранного $\varepsilon$ дающий $(2+\varepsilon)$-приблизительное решение DSA.
Более того, для некоторых частных случаев авторы предоставляют приближённую схему полиномиального времени (то есть $(1+\varepsilon)$-приближение).
Из них в рамках графа кадра особо интересна схема для случая ресурсов, размер которых ограничен сверху константой $h_{max}$.
Однако практичность представленных алгоритмов в рамках приложений реального времени является открытым вопросом в силу их высокой сложности, а также асимптотического характера теоретических оценок ошибки.

\section{Предложенное решение}
\subsection{Общая архитектура}
Кадровый граф в первую очередь состоит из вершин, каждая из которых в свою очередь состоит из имени и функций \textit{объявления} и \textit{запуска}, задаваемых пользователем рантайма.
Как следует из названия, функция запуска вызывается рантаймом в момент исполнения кадрового графа для записи командных списков, а функция объявления вызывается при перекомпиляции графа чтобы предоставить пользователю возможность установить следующие скрытые свойства вершины:
\begin{itemize}
    \item ресурсные и вершинные зависимости,
    \item назначения ресурсов,
    \item режим \textit{мультиплексирования} вершины,
    \item флаг наличия побочных эффектов,
    \item требуемое для запуска вершины состояние драйвера.
\end{itemize}
Пользователь может в любой момент создать, удалить или изменить вершину, на что рантайм автоматически отреагирует на ближайшем запуске кадрового графа.
Все существующие в программе вершины вершины организуются в единую структуру называемую кадровым графом, но в строгом математическом смысле кадровым графом не являющуюся.

Каждый кадр приложения пользователь даёт команду рантайму запустить текущий граф.
Перед тем как исполнить эту команду, рантайм проверяет не изменился ли граф с прошлого запуска, и в случае изменения начинает процесс \textit{компиляции графа}, состоящий из следующих пунктов:
\begin{enumerate}
    \item объявление вершин,
    \item генерация промежуточного представления,
    \item построение расписания вершин,
    \item построение расписания ресурсов,
    \item активация \textit{истории ресурсов}.
\end{enumerate}
Помимо вершин, кадровый граф содержит в себе некоторую вспомогательную информацию, при изменении которой нет нужды полностью перекомпилировать граф, поэтому процесс компиляции является инкрементальным.
Так, например, при изменении разрешении окна, процесс перекомпиляции графа начнётся с предпоследнего пункта.
Первый этап компиляции же запускается всякий раз когда пользователь изменил множество вершин, и именно в рамках этого этапа запускаются функции объявления изменившихся вершин.

Как уже было сказано выше, данные, изначально задаваемые пользователем рантайма, строго говоря не являются графом, а лишь схемой получения графа \textit{промежуточного представления}, с которым и работает остальной рантайм.
\todo{диаграмма?}
Данные, полученные от пользователя, могут быть невалидными и противоречивыми, поэтому в процессе получения промежуточного представления также идёт валидация, и гарантируется, что получившийся граф будет удовлетворять следующему определению.
\textit{Граф промежуточного представления}~--- пятёрка $(V, E, R, U, H)$, где $(V, E)$~--- вершины и рёбра ориентированного ациклического графа, $R$~--- множество ресурсов, функция $U : V \to 2^R$ задаёт множество используемых каждой вершиной ресурсов, а функция $H : V \to 2^R$ задаёт множество ресурсов, чью историю использует вершина.
При этом требуется, чтобы для любого $r\in R$ подграф $(V, E)$ индуцированный множеством $\left\{v \in V \middle| r \in U(v)\right\}$ был непустым слабо-связным графом с ровно одной вершиной с входной степенью $0$.

На этапе построения расписания вершин строится топологическая сортировка промежуточного графа.
В теории, именно на этом этапе может строится многопоточное расписание запуска функций исполнения вершин, а также могут выбираться разные очереди GPU для отправки команд исполняемыми вершинами.
Однако, на практике, работа в этом направлении требует крупных предварительных вложений времени в сугубо инженерные вопросы, что не было предусмотрено в рамках бюджета данной работы.

Эти и оставшиеся этапы, а также конкретные возможности разработанного решения, более подробно рассмотрены в следующих подразделах.

\subsection{Зависимости вершин}
Под ресурсами и вершинами как элементами множеств тут и в дальнейшем будем понимать их уникальные строковые идентификаторы, задаваемый пользователем рантайма.
У каждой вершины $v$, среди прочих, есть следующий набор скрытых свойств, устанавливаемых через функцию декларации:
\begin{itemize}
    \item множества предшествующих и последующих вершин $P_v$ и $F_v$,
    \item множество создаваемых ресурсов $C_v$,
    \item множество читаемых ресурсов $R_v$,
    \item множество изменяемых ресурсов $M_v$,
    \item множество пар переименования ресурсов $E_v$.
\end{itemize}
Именно на основании этих свойств рантайм генерирует рёбра в промежуточном графе по следующим правилам.
Ребро из вершины $v$ в вершину $u$ проводится тогда и только тогда, когда верно хотя бы одно из следующих утверждений:
\begin{itemize}
    \item $v\in P_u$,
    \item $u\in F_v$,
    \item $r \in C_v$ и $r \in M_u$,
    \item $r \in M_v$ и $r \in R_u$,
    \item $r \in R_v$ и $\exists r',\;(r, r') \in E_u$,
    \item $\exists r',\;(r', r) \in E_v$ и $r \in M_u$.
\end{itemize}
Иначе говоря, рантайм гарантирует, что создание ресурса произойдёт раньше, чем все модификации, каждая из модификаций произойдёт раньше, чем каждое чтение, и наконец каждое чтение произойдёт раньше, чем переименование.
При этом само переименование считается моментом создания нового ресурса, соответствующего новому имени.
Если в процессе генерации промежуточного представления из пользовательских вершин получается граф с циклами, либо какой-то ресурс создаётся более чем одной вершиной (считая переименования), рантайм оповещает пользователя об ошибке и предпринимает самостоятельную попытку исправить итоговый граф путём игнорирования некоторых вершин или рёбер.

При генерации промежуточного представления, информация о переименовывании стирается.
Так, одному ресурсу из множества $R$ промежуточного графа может соответствовать несколько различных имён ресурсов, указанных пользователем.

Подобная система упорядочивания гарантирует расширяемость и модульность приложений.Основной компонент приложения может составить \textit{скелетный} кадровый граф, не отображающий ничего, но задающий базовую структуру рендеринга.
\todo{схемка?}
Далее, различные модули приложения могут встраиваться в этот скелет, читая и модифицируя различные ресурсы.

\subsection{История ресурсов}
В процессе интеграции одной из ранних версий рантайма кадровых графов, было выяснено, что огромное количество алгоритмов визуализации в целевом приложении использует понятие \textit{истории ресурса}: требуется чтение данных конкретного ресурса в том виде, в котором они находились на конец предыдущего кадра.

Для поддержки таких алгоритмов было введено разделение на \textit{логические} и \textit{физические} ресурсы, где первые представляют собой строковые имена задаваемые пользователем рантайма, а последние~--- регионы памяти GPU, с которыми работают функции запуска вершин.
Для каждого логического ресурса кадрового графа создаётся два физических ресурса, предоставляемых пользователю поочерёдно на чётных и нечётных кадрах.
Это позволяет вершинам, представляющим такие алгоритмы как темпоральное сглаживание \cite{yang2020survey}, одновременно читать историю ресурса и писать сам ресурс.

В дополнение вышеперечисленным множествам запрашиваемых ресурсов каждая вершина содержит множество $H_v$ ресурсов, чья история требуется для выполнения вершины.
Запрос истории с точки зрения упорядочивания вершин приравнивается к чтению, что делает невозможным запрос истории ресурсов, переименуемых в процессе исполнения кадра.

Насколько известно автору, разработанное решение~--- первый рантайм кадровых графов, поддерживающий запросы истории ресурсов, а также способный переиспользовать память ресурсов, чья история запрашивается хотя бы одной вершиной.

\subsection{Мультиплексирование}
Следующая проблема, возникшая в процессе интеграции рантайма~--- необходимость запускать некоторые вершины кадрового графа несколько раз в рамках одного кадра.
Возникает такая необходимость в следующих ситуациях:
\begin{itemize}
    \item запуск приложения на устройстве виртуальной реальности, где вид с основной камеры необходимо рендерить для каждого дисплея заново (от 2 до 4 в зависимости от устройства);
    \item поддержка скриншотов в высоком разрешении, где скриншот рендерится по-частям, чтобы не превысить бюджет видеопамяти на потребительских GPU;
    \item поддержка алгоритма SSAA \todo{ссылка};
    \item поддержка локальной многопользовательской игры, где экран делится пополам, и на разных половинах отображается ракурс разных игроков.
\end{itemize}
Для поддержки этих ситуаций в предлагаемом решении введён механизм \textit{мультиплексирования} вершин и ресурсов.

Вводится размерность мультиплексирования, $D$.
Каждая вершина посредством функции декларации указывает свой \textit{режим мультиплексирования}, элемент $\mathbb{Z}_2^D$, являющийся булевой маской выбора измерений.
На режимах мультиплексирования вводится частичный порядок: для $a,b\in\mathbb{Z}_2^D$, $a \preceq b$, если $\forall i,\;a_i \leqslant b_i$.
Каждому ресурсу ставится в соответствие режим мультиплексирования той вершины, которая его создаёт.
Если вершина с режимом $a$ запрашивает ресурс с режимом $b$, то от пользователя требуется, чтобы $b \preceq a$, так как иная ситуация ведёт к неоднозначности выбора физического ресурса.
Также при $u \in P_v$ или $u \in F_v$ требуется, чтобы режимы $u$ и $v$ были сравнимы в частичном порядке $\preceq$.

Перед запуском графа, от пользователя требуется предоставить рантайму $c \in \mathbb{Z}_{>0}^D$.
Во время генерации промежуточного представления, для вершины с режимом $a$ будет создано $\prod_{i=1}^{D} c_i^{a_i}$ вершин  промежуточных промежуточного графа, и аналогично для ресурсов.
Правила проведения рёбер в промежуточном графе обобщаются со случая $c = \left(1, \dots, 1\right)$ естественным образом в силу ограничения из предыдущего абзаца.

Предлагаемая система мультиплексирования покрывает все описанные выше случаи.
Рассмотрим, например, приложение, работающее на VR-устройстве с двумя дисплеями, и использующее алгоритмом сглаживания SSAA x4
Выберем $D = 2$ и $c = (2, 4)$, где первое измерение будет отвечать номеру дисплея, а второе~--- количеству суперпикселей.
Все вершины приложения, визуализирующие вид с основной камеры, должны быть помечены режимом $(1, 1)$.
Вершины же, например, вычисляющие тени, должны быть помечены режимом $(0, 0)$, так как одна и та же карта теней может быть использована для обоих глаз и всех суперпикселей.

\subsection{Автоматические разрешения текстур}
Большие неудобства при разработке графических приложений составляет смена разрешения экрана.
Большинство транзиентных ресурсов представляют собой текстуры, разрешение которых кратно разрешению окна приложения, что делает целесообразным разработку централизованного механизма реакции на смену разрешения экрана.

Таким механизмом в предлагаемом решении являются \textit{авторазрешения}.
В вершине, создающей транзиентную текстуру, пользователь рантайма вместо конкретного разрешения может указать строковой идентификатор авторазрешения, числовое значение соответствующее которому сообщается рантайму извне, как правило в коде реакции на смену разрешения окна.
Рантайм, обнаружив на очередном кадре, что одно из авторазрешений поменялось, построит новое расписание ресурсов, тем самым изменяя разрешения физических ресурсов прозрачно для пользователя.

Также система авторазрешений предлагаемого решения поддерживает \textit{динамическую смену разрешений}.
Обнаружив низкую частоту кадров, приложение может уменьшить разрешение, в котором рисуются кадры, чтобы увеличить производительность.
Рантайм кадровых графов в свою очередь позволяет уменьшать разрешения всех текстур с производительностью достаточной чтобы делать это каждый кадр, не освобождая при этом, однако, неиспользуемую память.

\subsection{Расстановка барьеров}
Как уже было упомянуто ранее, современные графические API требуют от пользователя в ручную отслеживать использование ресурсов и расставлять в соответствии со спецификацией необходимые ресурсные барьеры.

В разработанном решении для каждого ресурса, встречающегося в множествах $C_v,$ $R_v$, $M_v$, $E_v$ и $H_v$, в вершине хранится пометка о том, как именно в рамках её функции запуска будет использоваться ресурс
Эта информация сохраняется при переходе к промежуточному представлению, и в момент построения расписания ресурсов рантайм в соответствии с этими пометками расставляет так называемые \textit{раздельные барьеры} \cite[раздел~7.5]{VulkanSpec}, позволяющие GPU самостоятельно выбирать подходящий момент времени для выполнения служебной работы по управлению кешами и состояниями ресурсов.

Как будет видно в последующих разделах, необходимость указывать способ использования каждого ресурса в каждой вершине не доставляет больших неудобств пользователям благодаря разработанному публичному API рантайма.

\subsection{Однородная поддержка ресурсов CPU}
\todo{блобы}

\subsection{Прореживание и валидация кадрового графа}
\todo{про прунинг, удаление неволидных ветвей, опциональные ресурсы}

\subsection{Управление глобальным состоянием}
\todo{про биндинг шейдерваров, матриц, рендертаргетов}

\subsection{Пользовательское API рантайма}
\todo{тут отрывочки кода, как оно выглядит}

\subsection{Построение расписания ресурсов}
\todo{тут мясо про CDSA}

\section{Результаты}
В процессе выполнения данной работы в рамках движка Dagor было реализовано предложенное решение в виде библиотеки рантайма кадровых графов <<daBFG>>, занимающей 13828 строк кода на языке C++, а также была произведена частичная интеграция этой библиотеки в существующие проекты, основанные на этом движке.
Так как масштаб кода отрисовки этих проектов невероятно велик, интеграция производится итеративно.
В результате изначальной интеграции, проведённой автором работы, было выделено около 30 вершин и порядка 10 ресурсов.
С тех пор к усилиям по интеграции присоединились и другие разработчики, и на данный момент в проекте Enlisted на максимальных настройках кадровый граф состоит из 84 вершин и отслеживает 74 ресурса, 27 из которых расположены в видеопамяти, управляемой самим рантаймом (остальные "--- CPU-ресурсы и внешний ресурсы).
Отметим, что более чем у половины из имеющихся ресурсов запрашивается история хотя бы одной вершиной.
Также стоит заметить, что некоторые проекты поддерживают работу в VR, что удваивает количество физических ресурсов, а также один из проектов поддерживает скриншоты в высоком качестве, что за счёт использования SSAA x4 и увеличения разрешения скриншота в 2 раза, посредством мультиплексирования увеличивает количество физических ресурсов в 64 раза.
Работа над интеграцией различных частей кода в кадровый граф продолжается, поэтому чтобы оценить качество разработанного решения управления памятью были использованы синтетические тесты.

\subsection*{Синтетические тесты}
В рамках работы над интеграцией решения в Dagor было замечено, что экземпляры задачи CDSA получаемые из временных ресурсов графических приложений обладают определённой структурой.
Например, б\'ольшая часть ресурсов является текстурами, чьё разрешение кратно разрешению пользовательского монитора.
Также, может показаться, что запрос истории ресурса "--- не самая частая операция.
Однако, как уже было сказано выше, среди ресурсов, чья память управляется кадровым графом, таких, чья история запрашивается, на данный момент больше половины.
Чтобы учесть эту специфику в синтетических тестах, был применён статистический метод \textit{бутстрэпа}.

Метода бутстрэпа заключается в генерации повторной выборки б\'ольшего размера из эмпирического распределения имеющейся небольшой выборки.
В качестве выборки были выбраны следующие параметры: количество моментов времени $T$, времена жизни ресурсов $\left|\left[l_i, r_i\right)\right|$ и размеры ресурсов $s_i$.
Также предполагается, что эти величины независимы от значений $l_i$, а сами величины $l_i$ распределены равномерно по всем моментам времени.
Эмпирическое распределение этих данных было получено из нескольких проектов, основанных на Dagor, на различных настройках качества графики.

В качестве базового метода был выбран распространённый в других решениях подход к обработке запросов истории ресурсов: ресурсы, история которых запрашивается хоть единажды, хранятся отдельно от всех остальных, и их память никогда не переиспользуется, а для остальных ресурсов решается задача DSA посредством жадного аллокатора реального времени.

Метрика для сравнения алгоритмов была заимоствована из предыдущих работ по приближённому решению задачи DSA, $makespan/LOAD$, где \textit{общая нагрузка} определяется как максимум нагрузок по всем моментам времени, $LOAD = \max_t L(t)$.
Эта метрика хорошо подходит для исследования DSA, так как $LOAD$ служит нижней границей для $makespan$, хотя часто и недостежимой.
Заметим, что в случае CDSA, несложно привести пример последовательности входных данных задачи с увеличивающимся количеством ресурсов такой, что $makespan/LOAD \geqslant 3/2$ для каждого элемента последовательности, см. \cite{myarticle}.
На практике, однако, подобные эффекты не возникают.

Результаты сравнений представлены на рисунке \ref{fig:results} в виде зависимости метрики $makespan/LOAD$ от количества ресурсов, где для каждого количества ресурсов замеры производились на 2000 различных синтетических тестах.
\begin{figure*}
\centering
\begin{tikzpicture}
\begin{axis}[
    width=0.9\textwidth,
    xlabel={Количество ресурсов},
    ylabel=$makespan/LOAD$,
    legend pos=north west,
    ymin=1,
    ymax=1.6,
    xmin=0,
    xmax=1650
]
\addlegendentry{Среднее, предложенный метод}
\addplot[mark=none,red] table[x=cnt, y=avg, col sep=comma]{results.csv};
\addlegendentry{90я перцентиль, предложенный метод}
\addplot[mark=none,green] table[x=cnt, y=p90, col sep=comma]{results.csv};
\addlegendentry{Среднее, базовый метод}
\addplot[mark=none,blue] table[x=cnt, y=badavg, col sep=comma]{results.csv};
\addlegendentry{90я перцентиль, базовый метод}
\addplot[mark=none,black] table[x=cnt, y=badp90, col sep=comma]{results.csv};
\end{axis}
\end{tikzpicture}
\caption{Замеры качества предложенной схемы решения CDSA по метрике $makespan/LOAD$, меньше "--- лучше.
Значения аккумулированы по 2000 запускам на синтетических тестах, полученных из реальных данных методом бутстрэпа.}
\label{fig:results}
\end{figure*}
%
\begin{figure*}
\centering
\begin{tikzpicture}
\begin{axis}[
    width=0.9\textwidth,
    height=0.3\textheight,
    xlabel={Количество ресурсов},
    ylabel=$makespan/LOAD$,
    legend pos=north east,
    ymin=1.09,
    ymax=1.175,
    xmin=100,
    xmax=1600,
]
\addlegendentry{Среднее, предложенный метод}
    \addplot[mark=none,red] table[x=cnt, y=avg, col sep=comma]{results.csv};
\addlegendentry{90я перцентиль, предложенный метод}
    \addplot[mark=none,green] table[x=cnt, y=p90, col sep=comma]{results.csv};
%
\addplot[black, thick] table[
    col sep=comma,
    y={create col/linear regression={y=avg}}
    ] {results_truncated.csv};
\addplot[black, thick] table[
    col sep=comma,
    y={create col/linear regression={y=p90}}
    ] {results_truncated.csv};
\end{axis}
\end{tikzpicture}
\caption{Графики предлагаемого решения из рис. \ref{fig:results} в увеличенном масштабе.}
\label{fig:resultsTrend}
\end{figure*}
Из рисунка \ref{fig:results} видно, что базовый метод ведёт к росту метрики с количеством ресурсов как в среднем так и в худшем случае.
Предложенное решение же приводит к убыванию метрики к асимптоте в районе 1.1, как в среднем, так и в худшем случае, что лучше видно на рисунке \ref{fig:resultsTrend}.
Подобный тренд полагает судить, что чем больше временных ресурсов движка Dagor интегрировано в кадровый граф, тем меньше будет управление их памятью отличатся от оптимального.
Оптимальное же отношение $makespan/LOAD$ находится в интервале $[1, 1.1]$, но нахождение этого оптимума не представляется возможным в силу $\NP$-полноты задачи.

Наконец стоит отметить, что предложенное решение не гарантирует ограниченности метрики $makespan/LOAD$ в асимптотическом смысле.
Иначе говоря, алгоритм имеет неограниченную ошибку относительно оптимального ответа.
Достаточно рассмотреть последовательность входов размера $n$ следующего вида, $T = n + 2$, $l_i = i$, $r_i = i + 2$, $i = \overline{0,n}$, при этом такую, что $\forall j,\;\sum_{i=0}^{j-2} s_i < s_j$ и $s_{j-1} < s_j$.
По индукции, при поиске блока для очередного ресурса $i$ будет доступен лишь единственный блок размера $\sum_{j=1}^{i-2} s_j$, по построению не достаточный для размещения ресурса $i$.
Следовательно, ресурсы будут расположены в памяти последовательно, а итоговый ответ $makespan = \sum_{i=0}^n s_i$.
Однако если рассмотреть эквивалентный экземпляр CDSA, в котором время повёрнуто вспять, алгоритм найдёт решение с $makespan = s_n + s_{n-1} = LOAD$, являющееся оптимальным.
Отношение $\left(\sum_{j=1}^{n} s_j\right)/\left(s_n + s_{n-1}\right)$ можно сделать сколь угодно большим, выбрав подходящее значение $s_i$, например, $2^{2^i}$, а значит и ошибка алгоритма не ограничена.

\section{Заключение}
В рамках данной работы был составлен обзор существующих решения, разработана архитектура рантайма кадровых графов, написана программная реализация предлагаемой архитектуры, произведена интеграция разработанного программного решения в коммерческие проекты, вопрос управления памяти был сведён к не рассмотренной ранее в литературе задаче дискретной оптимизации CDSA, предложена схема приближённого решения этой задачи, и проанализированы качество и скорость работы предложенной схемы.
Предложенное решение управления памятью позволяет сэкономить близкое к оптимальному количество памяти на данных, приближенных к реальным.

В рамках дальнейших исследований, во-первых, планируется проверка следующей гипотезы.
Ясно, что для некоторых промежуточных графов, выбранная топологическая сортировка влияет на величину $LOAD$.
Из этого резонно поставить вопрос о нахождении оптимальной топологической сортировки, минимизирующей $LOAD$, а значит и позволяющей уменьшить размер итогового ответа $makespan$.
Во-вторых, имеет смысл обобщение алгоритмов решения DSA предложенных в работах Йордана Гергова \cite{gergov_approximation_1996, gergov_algorithms_1999} на случай CDSA.
В-третьих, крупной, но малоизученной областью развития кадровых графов является автоматическая параллелизация вычислений как на CPU, так и на GPU.
Наконец, объединяя все предыдущие пункты, дальнейший интерес представляет возможность адаптации алгоритмов оптимизации кадрового графа под специфику конкретных устройств. Так, на TBDR-GPU для производительности графа в целом лучше группировать барьеры воедино (см. \cite{GraniteBlogPost}), сортируя при этом граф соответствующим образом, в то время как на дискретных GPU барьеры, как правило, слабее влияют на производительность, что может позволить выбрать иную сортировку графа, способствующую уменьшению $LOAD$. Параллельное исполнение графа также зависит от специфики устройства, количества ядер, количества модулей обработки списков команд на GPU, из чего можно полгать, что, вообще говоря, для различных устройств может быть оптимальным различное параллельное расписание исполнения вершин.

Кадровые графы "--- техника, совсем недавно получившая распространение в области компьютерной графики, но в перспективе способная улучшить все аспекты разработки приложений реального времени: потребление памяти, скорость работы приложения, и даже удобство разработки. Автор считает, что дальнейшие исследования, а также тяжкий труд по реализации теоретических идей на практике, позволят раскрыть этот потенциал и однажды сделать разработку высокопроизводительных приложений реального времени простым и приятным занятием.


\newpage

\emergencystretch=3em
\printbibliography
\end{document}

